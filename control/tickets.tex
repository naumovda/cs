\documentclass[oneside, final, 14pt]{extreport}
\usepackage[utf8]{inputenc}
\usepackage[russian]{babel}
\usepackage{vmargin}
\setpapersize{A4}
\setmarginsrb{2cm}{1cm}{1cm}{1cm}{0pt}{0mm}{0pt}{13mm}
\usepackage{indentfirst}
\usepackage{amsmath}
\usepackage{graphicx}

\newcounter{ticket}
\setcounter{ticket}{1}

\newcommand{\signline}{\_\_\_\_\_\_\_\_\_}

\newcommand{\cutline}{\_\_\_\_\_\_\_\_\_\_\_\_\_\_\_\_\_\_\_\_\_\_\_\_\_\_\_\_\_\_\_\_\_\_\_\_\_\_\_\_\_\_\_\_\_}

\begin{document}

% Билет 1
\input{header.tex}
\begin{enumerate}
\item Трехуровневая архитектура СУБД.
\item SQL. Разделы SQL.
\item Операционная система. Определение.
\end{enumerate}
\input{bottom.tex}

\cutline

% Билет 2
\input{header.tex}
\begin{enumerate}
\item Иерархическая модель. XML как пример реализации иерархической модели данных. Преимущества и недостатки иерархической модели данных.
\item MySQL как реляционная СУБД. Типы данных. NULL-значения.
\item Структура операционной системы.
\end{enumerate}
\input{bottom.tex}

\newpage

% Билет 3
\input{header.tex}
\begin{enumerate}
\item Сетевая модель данных. Понятия "запись", "набор данных". Преимущества и недостатки сетевой модели данных.
\item SQL: оператор Select.
\item Слои программного обеспечения в компьютерной системе.
\end{enumerate}
\input{bottom.tex}

\cutline

% Билет 4
\input{header.tex}
\begin{enumerate}
\item Основные понятия реляционной модели данных: отношение, кортеж.
Свойства отношения. Схема отношения. Преимущества и недостатки реляционной модели данных.
\item SQL: оператор Update.
\item Операционная система как виртульная машина.
\end{enumerate}
\input{bottom.tex}

\newpage

% Билет 5
\input{header.tex}
\begin{enumerate}
\item Прототипирование. Виды прототипов: момент-интервал, роль, группа.
\item SQL: оператор Delete.
\item Операционная система как менеджер ресурсов.
\end{enumerate}
\input{bottom.tex}

\cutline

% Билет 6
\input{header.tex}
\begin{enumerate}
\item Уровни абстракции данных. Концептуальный уровень. Логический уровень. Физический уровень.
\item Системы управления версиями. Локальная СУВ.
\item Защита пользователей и программ. 
\end{enumerate}
\input{bottom.tex}

\newpage

% Билет 7
\input{header.tex}
\begin{enumerate}
\item Тип сущности. Экземпляр сущности. Степень типа связи. Ролевые имена сущностей. Рекурсивные связи.
\item Централизованная СУВ.
\item Понятие системного вызова.
\end{enumerate}
\input{bottom.tex}

\cutline

% Билет 8
\input{header.tex}
\begin{enumerate}
\item Атрибуты. Домены. Однозначные и многозначные атрибуты. Вычисляемые атрибуты. Атрибуты связи.
\item Децентрализованная СУВ.
\item Прерывание. Исключительная ситуация.
\end{enumerate}
\input{bottom.tex}

\newpage

% Билет 9
\input{header.tex}
\begin{enumerate}
\item Потенциальный ключ. Первичный ключ. Составной ключ.
\item Git. Состояния файлов в Git.
\item Файл. Файловая система.
\end{enumerate}
\input{bottom.tex}

\cutline

% Билет 10
\input{header.tex}
\begin{enumerate}
\item Правила перехода от концептуальной модели данных к реляционной.
\item Основные понятия ветвления в Git.
\item Классификация операционных систем.
\end{enumerate}
\input{bottom.tex}

\end{document}
